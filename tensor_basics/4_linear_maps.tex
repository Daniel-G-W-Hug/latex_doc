\subsection{Linear Maps}

Linear maps take a vector as an input and produce a new vector as an output. Linear maps
don't transform the basis vectors. They just transform the input vectors into the output
vectors given in the same basis.\\

Linear maps shown as matrices map copies of the unit basis vectors $[1,0]^T$ and $[0,1]^T$
into the vectors their coordinate columns consist of (see matrix multiplication with the
corresponding vector from the right side). The first column is the mapped value of the
first basis vector and the second column the mapped value of the second basis vector. In
general the $i^{th}$ column is the mapped value of the $i^{th}$ basis vector.\\

From a geometrical point of view linear maps are spacial transforms that keep gridlines
parallel, keep gridlines evenly spaced, and keep the origin stationary. That means linear
maps can stretch and rotate the input vectors, but they don't shift the origin.\\

From an abstract point of view linear maps do map vectors to vectors and have the
linearity properties, i.e. can be added and scaled. That means:
\begin{equation}
    \label{eq:linear_map_properties}
    \begin{array}{rcl}
        L: V \rightarrow  V &\text{or}& L: V \rightarrow W \\
        L(\textcolor{YellowOrange}{\vec{v}} + \textcolor{RoyalPurple}{\vec{w}}) & = &
        L(\textcolor{YellowOrange}{\vec{v}}) + L(\textcolor{RoyalPurple}{\vec{w}}) \\
        L(\textcolor{ForestGreen}{n}\textcolor{YellowOrange}{\vec{v}}) & = &
        \textcolor{ForestGreen}{n}L(\textcolor{YellowOrange}{\vec{v}}) 
    \end{array}
\end{equation}

We can derive the linear map and the resulting matrix multiplication of the vector
components directly from the linearity properties~(\ref{eq:linear_map_properties}):
\begin{equation}
    \label{eq:linear_map_derive_matrix_multiplication}
    \begin{array}{rcl}
        \textcolor{RoyalPurple}{\vec{w}} = L(\textcolor{YellowOrange}{\vec{v}}) & = &
        L(\hdvc{1}\hdbv{1} + \hdvc{2}\hdbv{2}) \\
        & \underset{(\ref{eq:linear_map_properties})}{=} &
        \hdvc{1}L(\hdbv{1}) + \hdvc{2}L(\hdbv{2})
    \end{array}
\end{equation}

Assuming $L: V\rightarrow V$, we can express the resulting vector in
\{\hdbv{1},\hdbv{2}\}:
\begin{equation}
    \label{eq:linear_map_derive_matrix_multiplication_2}
    \begin{array}{rcl}
        L(\hdbv{1}) & = & L^{1~}_{~1}\hdbv{1} + L^{2~}_{~1}\hdbv{2} \\
        L(\hdbv{2}) & = & L^{1~}_{~2}\hdbv{1} + L^{2~}_{~2}\hdbv{2} \\
        L(\hdbv{j}) & = & L^{k~}_{~j}\hdbv{k}
    \end{array}
\end{equation}
%Using~(\ref{eq:linear_map_derive_matrix_multiplication_2}) results in:
\begin{equation}
    \label{eq:linear_map_derive_matrix_multiplication_3}
    \begin{array}{rcl}
        \Rightarrow\textcolor{RoyalPurple}{\vec{w}} =
        L(\textcolor{YellowOrange}{\vec{v}}) & = &
        \hdvc{1}(L^{1~}_{~1}\hdbv{1} + L^{2~}_{~1}\hdbv{2}) +
        \hdvc{2}(L^{1~}_{~2}\hdbv{1} + L^{2~}_{~2}\hdbv{2}) \\
        & = & (L^{1~}_{~1}\hdvc{1}+ L^{1~}_{~2}\hdvc{2})\hdbv{1} +
        (L^{2~}_{~1}\hdvc{1} + L^{2~}_{~2}\hdvc{2})\hdbv{2} \\
        & = & \hspace{35pt}w^1\hspace{30pt}\hdbv{1} +
        \hspace{35pt}w^2\hspace{30pt}\hdbv{2} \\
        \noalign{\vskip10pt}
        w^1 & = & L^{1~}_{~1}\hdvc{1}+ L^{1~}_{~2}\hdvc{2} \\
        w^2 & = & L^{2~}_{~1}\hdvc{1} + L^{2~}_{~2}\hdvc{2} \\
        \noalign{\vskip10pt}
        \Rightarrow w^i & = &
        L^{i~}_{~j}\hdvc{j}
    \end{array}
\end{equation}


To figure out the transformation rules for linear maps we need to express the linear map
as an expansion in the new basis to see how the new basis vector are mapped:
\begin{equation}
    \label{eq:linear_map_transform_to_new_basis}  
    \begin{array}{rcl}
        \widetilde{L^{l~}_{~i}}\hdbtv{l} & = & L(\hdbtv{i}) \\
        & \underset{(\ref{eq:forward_trafo})}{=} &
        L(F^{j~}_{~i}\hdbv{j}) \underset{(\ref{eq:linear_map_properties})}{=}
        F^{j~}_{~i}L(\hdbv{j}) \\
        & \underset{(\ref{eq:linear_map_derive_matrix_multiplication_2})}{=} &
        F^{j~}_{~i}L^{k~}_{~j}\hdbv{k} \underset{(\ref{eq:backward_trafo})}{=}
        F^{j~}_{~i}L^{k~}_{~j} B^{l~}_{~k}\hdbtv{l} \\
        \widetilde{L^{l~}_{~i}}\hdbtv{l} & = &
        B^{l~}_{~k} L^{k~}_{~j} F^{j~}_{~i} \hdbtv{l} \\
        \noalign{\vskip10pt}
        \Rightarrow \textcolor{red}{\widetilde{L^{l~}_{~i}}} & = &
        B^{l~}_{~k} \textcolor{MidnightBlue}{L^{k~}_{~j}} F^{j~}_{~i}
    \end{array}
\end{equation}

To figure out the transformation to the old basis we can use the fact that the forward and
the backward transform are inverses of each other~(\ref{eq:forward_backward_inverse}). By
multiplying~(\ref{eq:linear_map_transform_to_new_basis}) from the left with the forward
transformation matrix $F^{s~}_{~l}$ and from the right with the backward transformation
matrix $B^{i~}_{~t}$we get:
\begin{equation}
    \label{eq:linear_map_transform_to_old_basis}  
    \begin{array}{rcl}
        \widetilde{L^{l~}_{~i}} & = &
        B^{l~}_{~k} L^{k~}_{~j} F^{j~}_{~i} \\
        F^{s~}_{~l} \widetilde{L^{l~}_{~i}} B^{i~}_{~t} & = &
        F^{s~}_{~l} B^{l~}_{~k} L^{k~}_{~j} F^{j~}_{~i} B^{i~}_{~t} \\
        & = & \quad \delta^s_k \quad L^{k~}_{~j} \quad \delta^j_t \quad = L^{s~}_{~t} \\
        \noalign{\vskip10pt}
        \Rightarrow \textcolor{MidnightBlue}{L^{s~}_{~t}} & = &
        F^{s~}_{~l} \textcolor{red}{\widetilde{L^{l~}_{~i}}} B^{i~}_{~t}  
    \end{array}
\end{equation}

Linear transformations can also be interpreted as linear combinations of vector-covector
pairs, i.e. as sum of tensor products of covariant basis vectors $\hdbv{i}$ and
contravariant covector basis vectors $\hdcbvc{j}$. They form a basis for any arbitrary
linear transformation matrix, i.e. the set
$\{\hdbv{1}\hdcbvc{1},\hdbv{1}\hdcbvc{2},\hdbv{2}\hdcbvc{1},\hdbv{2}\hdcbvc{2}\}$ is a
basis for $V \rightarrow V$ (the same consideration is possible for the new basis).
\begin{equation}
    \begin{array}{rcl}
        \label{eq:linear_map_def_as_vec-covec} 
        L & = & a\hdbv{1}\hdcbvc{1}+
                b\hdbv{1}\hdcbvc{2}+
                c\hdbv{2}\hdcbvc{1}+
                d\hdbv{2}\hdcbvc{2} \\
        L & = & L^{i~}_{~j} \hdbv{i}\hdcbvc{j} =
        L^{i~}_{~j} (\hdbv{i} \otimes\hdcbvc{j})
    \end{array}
\end{equation}
In matrix notation the matrix for an arbitrary linear transformation is:
\begin{equation}
    a\begin{bmatrix} 1 & 0 \\ 0 & 0 \end{bmatrix} + 
    b\begin{bmatrix} 0 & 1 \\ 0 & 0 \end{bmatrix} + 
    c\begin{bmatrix} 0 & 0 \\ 1 & 0 \end{bmatrix} + 
    d\begin{bmatrix} 0 & 0 \\ 0 & 1 \end{bmatrix} =
    a\begin{bmatrix} a & b \\ c & d \end{bmatrix}
\end{equation}

Using the second equation of (\ref{eq:linear_map_def_as_vec-covec}) and the vector
definition $\vec{v} = \hdvc{k} \hdbvc{k}$ we get:
\begin{equation}
    \begin{array}{rcl}
        \vec{w} & = & L(\vec{v}) \\
          & =  & L^{i~}_{~j} \hdbv{i}\hdcbvc{j} (\hdvc{k} \hdbvc{k}) \\
          & = & L^{i~}_{~j} \hdvc{k}\hdbv{i}\hdcbvc{j}(\hdbvc{k}) \\
          & = & L^{i~}_{~j} \hdvc{k}\hdbv{i} \delta^j_k \\
          \noalign{\vskip10pt}
        \Rightarrow \vec{w} & = & L^{i~}_{~j} \hdvc{j}\hdbv{i} \\
        w^i & = & L^{i~}_{~j} \hdvc{j}
    \end{array}
\end{equation}

\newpage
