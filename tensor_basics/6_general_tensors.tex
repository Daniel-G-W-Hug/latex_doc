\subsection{Transformation laws of arbitrary tensors}
Let's review the defintion of a tensor: \emph{``A tensor is an object that is invariant under a
a change of coordinates, and has components that change in a special, predictable way
under a change of coordinates.''} \\

\textbf{To sum up everything up to here:}\\

\textcolor{red}{\underline{Contra}variant (1,0)-tensors} transform like this (basis /
components) - the components can be written as column vectors and are multiplied to the
matrix from the right:
\begin{equation}
    \begin{array}{rcl}
        \hdcbtvc{i} & = & B^{i~}_{~j} \hdcbvc{j} \\
        \hdcbvc{i} & = & F^{i~}_{~j}  \hdcbtvc{j}
    \end{array}
    \qquad
    \begin{array}{rcl}
        \hdtvc{i} & = & B^{i~}_{~j} \hdvc{j} \\
        \hdvc{i} & = & F^{i~}_{~j}  \hdtvc{j}
    \end{array}
\end{equation}

\textcolor{red}{\underline{Co}variant (0,1)-tensors} transform like this (basis /
components) - the components can be written as row vectors and are multiplied to the
matrix from the left:
\begin{equation}
    \begin{array}{rcl}
        \hdbtv{j} & = & \hdbv{i}  F^{i~}_{~j} \\
        \hdbv{j} & = & \hdbtv{i} B^{i~}_{~j} 
    \end{array}
    \qquad
    \begin{array}{rcl}
        \widetilde{{\alpha_j}} & = & {\alpha_i} F^{i~}_{~j}  \\
        {\alpha_j} & = & \widetilde{{\alpha_i}} B^{i~}_{~j}
    \end{array}
\end{equation}

\textcolor{red}{Linear maps are (1,1)-tensors}, i.e. they have a vector and a covector
part and transform like this - the matrix of the coefficients of the linear map is
multiplied from the left and the right with the respective forward or backward
transformation matrices:
\begin{equation}
    \begin{array}{rcl}
        \textcolor{red}{\widetilde{L^{i~}_{~j}}} & = &
        B^{i~}_{~k} \textcolor{MidnightBlue}{L^{k~}_{~l}} F^{l~}_{~j} \\
        \textcolor{MidnightBlue}{L^{i~}_{~j}} & = &
        F^{i~}_{~k} \textcolor{red}{\widetilde{L^{k~}_{~l}}} B^{l~}_{~j}
    \end{array}
\end{equation}

\textcolor{red}{Metric tensors are (0,2)-tensors}, i.e. they have two covector parts and
transform like this:
\begin{equation}
    \begin{array}{rcl}
        \textcolor{red}{\tilde{g}_{ij}} & = &
        F^{k~}_{~i}  F^{l~}_{~j} \textcolor{MidnightBlue}{{g}_{kl}} \\
        \textcolor{MidnightBlue}{g_{kl}} & = &
        B^{i~}_{~k} B^{j~}_{~l} \textcolor{red}{\tilde{g}_{ij}}
    \end{array}
\end{equation} \\

\textbf{The general transformation laws for tensor components for an arbitray tensor of type
$(m,n)$ with $m$ upstairs indices and $n$ downstairs indices are:}
\begin{equation}
    \begin{array}{rcl}
        \textcolor{red}{\widetilde{T}^{abc...}_{xyz...}} =
            (B^{a~}_{~i} B^{b~}_{~j} B^{c~}_{~k} \cdots)
            \,\textcolor{MidnightBlue}{T^{ijk...}_{rst...}}\,
            (F^{r~}_{~x} F^{s~}_{~y} F^{t~}_{~z} \cdots) \\
        \noalign{\vskip10pt}
        \textcolor{MidnightBlue}{T^{ijk...}_{rst...}} =
            (F^{i~}_{~a} F^{j~}_{~b} F^{k~}_{~c} \cdots)
            \,\textcolor{red}{\widetilde{T}^{abc...}_{xyz...}}\,
            (B^{x~}_{~r} B^{y~}_{~s} B^{z~}_{~t} \cdots)
    \end{array}
\end{equation}
The upstairs indices represent the contravariant components of the tensor and the
downstairs indices represent the covariant components of the tensor. The transformation
itself is a series of backward and forward transforms. To go from old to new system the
covariant componens use forward transformations, while the contravariant components use
backward transformations. \\

Anything that follows these transformation rules during change of coordinates is a tensor.

\newpage