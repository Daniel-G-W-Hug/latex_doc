\documentclass[12pt,twoside,a4paper]{article}
% Umlaute direkt nutzen (Text muss direkt in utf8 vorliegen => Default bei sublime):
% statt \"A, \"O, \"U, \"a, \"o, \"u und \ss{}
% direkt Ä, Ö, Ü, ä, ö, ü und ß im Text engeben
\usepackage[utf8]{inputenc}  % Umlaute direkt eingeben (ggf. latin1 statt utf8)
\usepackage[T1]{fontenc}     % Ausgabe der Umlaute
\usepackage{lmodern}

\usepackage[dvips]{graphicx}       % alles zum Grafiken einbinden

\usepackage{eurosym}               % Eurosymbol

\usepackage[centertags]{amsmath}   % Mathekram: amstex
\usepackage{amssymb}
\usepackage{latexsym}              % math symbols
\usepackage{exscale}               % Summen-/Integralzeichen in richtiger Groesse

% Deutsch als Hauptsprache im Dokument (Layout von Datum, Trennungsregeln etc.)
% Umschaltbar mit \selectlanguage{}
\usepackage[main=ngerman, english]{babel}  % support "Neue Rechtschreibung"

\usepackage[version=4]{mhchem}     % Chemische Formeln (muss nach amsmath kommen!)


\begin{document}

% \title{Mein Titel}
% \author{Mein Name hier}
% \date{\today}
% \maketitle
% \thispagestyle{empty}
% \newpage

\section{Erster Abschnitt}
Etwas Text. Und der ganze Spaß kostet \euro{25}. Und sogar Chemiekram geht: \ce{C6H12O6 + 6O2 -> 6CO2 + 6H2O} oder \ce{Sb2O3} oder \ce{^227_90Th+}.


\section{Zweiter Abschnitt}
Umlaute als Steuerzeichen: \"A, \"O, \"U, \"a, \"o, \"u und \ss{} \\
Umlaute direkt: Ä, Ö, Ü, ä, ö, ü und ß dank Paket "`inputenc"'\\

Meine erste Formel:
\begin{equation}
\text{Seien } a,b \in \mathbb{R}, \text{ dann gilt } (a+b)^{2} = a^{2} + 2ab + b^{2}
\end{equation}

Einfacher geht es auch so $(a+b)^{2} = a^{2} + 2ab + b^{2}$

\section{Dritter Abschnitt}

\begin{itemize} 
\item Ein Stichpunkt 
\item Noch ein Stichpunkt 
\end{itemize}

\begin{enumerate} 
\item Erster Stichpunkt
\item Zweiter Stichpunkt 
\end{enumerate}

\section{Abschnitt mit Bild}

\begin{figure}[htbp]
\centering
\includegraphics[width=0.2\textwidth]{Knut}
\caption{Unser neues Logo}
\label{fig:Bild1}
\end{figure}

\end{document}