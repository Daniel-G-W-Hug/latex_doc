\subsection{Basis Transformation Rules}

\begin{figure}[h]
    \centering
    \includegraphics[width=0.25\textwidth]{Old_new_base_eigenchris}
    \caption{old and new basis vectors}
    \label{fig:old_new_base}
\end{figure}

\emph{Basis transformation:} The basis vectors of the new system expressed as a linear
combination of the basis vectors of the old system is considered a forward transformation.
The other direction is considered a backward transformation.\\

Here $F$ stands for $\underline{F}$orward transformation. This transformation maps the
basis vectors of the old basis to the new basis vectors. The coordinates of the mapped
basis vectors in the old system become the columns of the resulting forward transformation
matrix. Basis vectors are chosen to be left multiplied to the transformation matrix, which
corresponds to summation via the first index of the transformation matrix. Indices are
chosen upper and lower to enable the Einstein summation convention later on.
Transformation indices are always from ``north west'' to ``south east'':
\begin{equation}
    \label{eq:forward_trafo}
    \begin{array}{rcl}
        [\hdbtv{1} \quad \hdbtv{2}] & = &
        [\hdbv{1} \quad \hdbv{2}]
        \begin{bmatrix}
            F^{1~}_{~1} & F^{1~}_{~2} \\
            F^{2~}_{~1} & F^{2~}_{~2}
        \end{bmatrix} \\
        \noalign{\vskip10pt}
        \hdbtv{1} & = & F^{1~}_{~1}\hdbv{1} + F^{2~}_{~1}\hdbv{2}
        \quad \underset{\text{figure}~\ref{fig:old_new_base}}{=} \quad
        2 \hdbv{1} + 1 \hdbv{2} \\
        \hdbtv{2} & = & F^{1~}_{~2}\hdbv{1} + F^{2~}_{~2}\hdbv{2}
        \quad \underset{\text{figure}~\ref{fig:old_new_base}}{=} \quad
        -\frac{1}{2}\hdbv{1} + \frac{1}{4}\hdbv{2} \\
        \noalign{\vskip10pt}
        \Rightarrow \hdbtvc{j} & = &
        F^{k~}_{~j} \hdbvc{k} \quad\text{(summation convention)}
    \end{array}
\end{equation}

Here $B$ stands for $\underline{B}$ackward transformation:
\begin{equation}
    \label{eq:backward_trafo}
    \begin{array}{rcl}
        [\hdbv{1} \quad \hdbv{2}] & = &
        [\hdbtv{1} \quad \hdbtv{2}]
        \begin{bmatrix}
            B^{1~}_{~1} & B^{1~}_{~2} \\
            B^{2~}_{~1} & B^{2~}_{~2}
        \end{bmatrix} \\
        \noalign{\vskip10pt}
        \hdbv{1} & = & B^{1~}_{~1}\hdbtv{1} + B^{2~}_{~1}\hdbtv{2}
        \quad \underset{\text{figure}~\ref{fig:old_new_base}}{=} \quad
        \frac{1}{4} \hdbtv{1} + (-1) \hdbtv{2}\\
        \hdbv{2} & = & B^{1~}_{~2}\hdbtv{1} + B^{2~}_{~2}\hdbtv{2}
        \quad \underset{\text{figure}~\ref{fig:old_new_base}}{=} \quad
        \frac{1}{2}\hdbtv{1} + 2 \hdbtv{2} \\
        \noalign{\vskip10pt}
        \Rightarrow \hdbvc{i} & = &
        B^{j~}_{~i}\hdbtvc{j}\quad\text{(summation convention)}
    \end{array}
\end{equation}

Between $B$ and $F$ following relation holds:
\begin{equation}
    \label{eq:forward_backward_inverse}
    \begin{array}{rcl}
        B^{j~}_{~i} F^{k~}_{~j} & = & F^{k~}_{~j} B^{j~}_{~i}
        = \delta^k_i =
        \begin{cases}
            1, & \text{if}\ i = k \\
            0, & \text{if}\ i \neq k
        \end{cases} \\
        \text{full equation transposed:} & = &
        B^{i~}_{~j} F^{j~}_{~k} = \delta^i_k \\
        \noalign{\vskip10pt}
        \Rightarrow B & = & F^{-1} \quad\text{($B$ is the inverse of $F$)}
    \end{array}
\end{equation}


\newpage
